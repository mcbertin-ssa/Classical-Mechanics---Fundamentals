We may always separate a system of particles, and form sub-systems, by using a definite criterium. For example, a system with $n\in\mathbb{N}$ particles, each with a mass $m_1$, and another system with $k\in\mathbb{N}$ particles, each with a mass $m_2$, may be seen as two sub-systems of a larger system with $n+k$ particles with distinct masses. \footnote{Here, $\mathbb{N}$ represents the set of natural numbers, which will be used to count properties of many particle systems.} In this case, a system with $n$ particles can always be seen as $n$ 